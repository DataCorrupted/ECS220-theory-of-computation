\subsection*{Existence, search, and the oracle. (Textbook problem 1.10)}
    \begin{quote}
    We can consider two rather different problems regarding Hamiltonian cycles (note a Hamiltonian \emph{cycle} must start and end at the same vertex). One is the decision problem, the yes-or-no question of whether such a cycle exists. The other is the search problem or function problem, in which we want to actually find the cycle. However, the hardness of these problems is very closely related.

    Suppose that there is an oracle in a nearby cave. She will tell us, for the price of one drachma per question, whether a graph has a Hamiltonian cycle. If it does, show that by asking her a series of questions, perhaps involving modified versions of the original graph, we can find the Hamiltonian cycle after spending a number of drachmas that grows polynomially as a function of the number of vertices. Thus if we can solve the decision problem in polynomial time, we can solve the search problem as well.
    \end{quote}

\solution

Intuitively we understand that a Hamiltonian cycle is deterministic, this means that it will not be affected by the existence of other edges.
In other words, if an edge is not in a Hamiltonian cycle, removing it will not affect the result.
However, if a graph has multiple Hamiltonian cycles (consider a 4-clique), we are free to destroy one cycle and the others will persist. 
The proof follows later.

Now we can present the following algorithm:

\begin{algorithm}
\caption{Get Hamiltonian Cycle with the help of an Oracle}
\begin{algorithmic}
	\REQUIRE $G = (E, V)$
	\STATE $H \leftarrow E$
	\FOR {$\forall e \in E$} 
		\IF {Oracle($G(H / \{e\}, V)$) == true} 
			\STATE $H \leftarrow H / \{e\}$
		\ENDIF
	\ENDFOR
	\RETURN $H$
\end{algorithmic}
\end{algorithm}

Now let's prove that this algorithm is correct:

\begin{proof}
Suppose a true Hamiltonian cycle is $H'$.

First, it is trivial to see that during the removal process, removing a non-Hamiltonian edge will not nullify previous Hamiltonian edge(s).

Therefore we can see that $G' = (H, V)$ have a Hamiltonian cycle, or else $H = \emptyset$ and $G$ has no Hamiltonian cycle(i.e. $H' = \emptyset$ too), leading to $H' \subset H$.

Now we want to prove that $H \subset H'$.

Suppose we have $h \in H, h \notin H'$. But this is impossible as it should have been removed in the course of executing our algorithm, since \texttt{Oracle($G(H / \{h\}, V)$) == true}.

So we have $H == H'$, the subset of edges we picked should be a Hamiltonian cycle.

\end{proof}

Now if the Oracle can solve this in $\P$, since we only do $O(|E|) = O(|V|^2)$ queries to, the overall time complexity is still in $\P$. 
