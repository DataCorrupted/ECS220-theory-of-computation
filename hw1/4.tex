\subsection*{Slightly faster algorithm for Hamiltonian path. (see Textbook problem 3.31)}
\begin{quote}
    We now consider the Hamiltonian Path decision problem (where a Hamiltonian path visits each vertex exactly once, but does not start and end at the same vertex).
    The na\"{i}ve brute-force algorithm will try to check all $n!$ length-$n$ sequences of unique nodes to see if they are valid paths in the graph.

    We can do slightly better using dynamic programming.

    Finally, show that $2^n \cdot \poly(n)=o(n!)$, so this is a better exponential-time algorithm than the na\"{i}ve brute-force algorithm. (Hint: Use Stirling's Approximation \url{https://en.wikipedia.org/wiki/Stirlings_approximation}).
\end{quote}

\begin{solution}

    The idea of the solution is here: for every subset $S$ of vertices, check whether there is a path that visits each node in $S$ exactly once and ends at a vertex $v$. Do this for all $v \in S$. If it exists, $v$ can be added to subset $S$ and we can say that $S$ has a Hamiltonian path. We can loop this action until we expand $S$ to contain all vertices in the graph, and we have our answer to the Hamiltonian Path decision problem. 
    
    We use bitmasking to represent subsets, so our algorithm is equivalent to algorithm \ref{code:dp}. Adjacency matrix $A$ is required.
    
    \begin{algorithm}
    	\caption{Hamiltonian Path decision problem} 
    	\label{code:dp}
        \begin{algorithmic}[1]
        \REQUIRE {A[ ][ ]}
    	\FOR {i = 0 to $2^n-1$}
        	\FOR {j = 0 to n}
        	    \STATE {dp[j][i] = false}
        	\ENDFOR
    	\ENDFOR
    	\FOR {i = 0 to n-1}
    	    \STATE {dp[i][$2^i$] = true}
    	\ENDFOR
    	\FOR {i = 0 to $2^n-1$}
    	    \FOR {j = 0 to n-1}
    	        \IF {$j^{th}$ bit is set in i}
    	            \FOR {k = 0 to n-1}
    	                \IF {j != k and $k^{th}$ bit is set in i and A[k][j] == true}
    	                    \IF {dp[k][i XOR $2^j$ ] == true}
    	                        \STATE {dp[j][i]=true}
    	                        \STATE {break}
    	                    \ENDIF
    	                \ENDIF
    	            \ENDFOR
    	        \ENDIF
    	   \ENDFOR
    	\ENDFOR
    	\FOR {i = 0 to n}
            \IF {dp[i][$2^n$-1] == true}
                \STATE {return true}
            \ENDIF
        \ENDFOR
        \STATE {return false}
        \end{algorithmic}
    \end{algorithm}

    This algorithm is capable of solving the decision problem in $O(2^n \times n^2)$ time with a space complexity of $O(n2^n)$. This algorithm does not address the Hamiltonian path search problem since it doesn't store the path or the history of visiting the nodes. 
    
    Next, we want to explain the algorithm a bit. Line 1 to line 5 initializes the $dp$ matrix to be all $false$. Since we use bitmasking to represent subsets, $dp[j][i]$ represents whether there is a qualified path in subset $S_i$ and end at node $V_j$. Line 6 to line 8 set subset $S_{2^i}$ to be true because $S_{2^i}$ contains only $V_i$, so it should have a qualified path. Line 9 to line 22 is to iterate all bitmasks(subsets) and all vertices to expand subset $S$, as stated above. Line 11 is to check whether vertex $V_j$ is in subset $S_i$. Line 13 is finds the neighbor of vertex $V_j$ in subset $S_i$, and line 14 checks whether $S_i - \{V_j\}$ has a qualified path or not. 
    
    To prove $2^n \times poly(n)=o(n!)$, we need to prove
    \begin{equation}
        \label{eqn:1}
        \lim_{n \to \infty} \frac{2^nn^m}{n!} = 0
    \end{equation}
    m is a random constant. We can prove this by proving
    \begin{equation}
        \label{eqn:2}
        \lim_{n \to \infty} log(\frac{2^nn^m}{n!}) = -\infty
    \end{equation}
    $$log(\frac{2^nn^m}{n!}) = n + mlogn - logn!$$
    By using Stirling's Approximation, it can be written as
    $$\lim_{n \to \infty} log(\frac{2^nn^m}{n!}) = n + mlogn - nlogn + nloge - O(logn) = -O(nlogn) = -\infty$$
    
    so we prove that equation \ref{eqn:2} is correct. Thus equation \ref{eqn:1} is correct and $2^n \times poly(n)=o(n!)$.

\end{solution}
