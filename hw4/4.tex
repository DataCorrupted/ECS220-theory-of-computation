\documentclass[11pt]{article}

%\usepackage[nosol]{optional}
\usepackage[sol]{optional}
\usepackage{listings, ../listings-rust/listings-rust}

\newcommand{\N}{\mathbb{N}}
\newcommand{\Z}{\mathbb{Z}}
\renewcommand{\P}{\mathsf{P}}
\newcommand{\NP}{\mathsf{NP}}
\newcommand{\SAT}{\textsc{Sat}}

\usepackage{enumerate,color,comment}
\usepackage{graphicx}

% For proof.
\usepackage{amsmath,amsthm,amssymb}

\usepackage{fullpage,amsmath,amssymb}
\usepackage[colorlinks=true,citecolor=blue,linkcolor=blue]{hyperref} % for href links, and also makes \ref and \eqref clickable in the PDF


\usepackage{fancyhdr}
\fancypagestyle{firststyle}
{
   \fancyhf{}
   \fancyhead[C]{Copyright \copyright\ \today, David Doty}
}


\title{Homework 4 \opt{sol}{Solutions} -- ECS 220, Winter 2020}
\date{}
\newtheorem{theorem}{Theorem}
\begin{document}
\maketitle
\thispagestyle{firststyle}
\vspace{-2.0cm}


\section{Two dimensions.}
    Recall that a counter machine is like a finite-state robot that lives in the discrete first quadrant, with infinite walls along the $x$-axis and $y$-axis,
    that can detect if it is up against either wall, but otherwise has no information about its position.
    Now we generalize this to other types of regions.
    \begin{quote}
    Consider two-dimensional finite automata that are trapped inside some finite region in the planar grid.
    At each step, they can move one step in one of the cardinal directions based on their current state, and then update their state.
    Like the automata discussed in Section 7.6.1, they cannot write anything on the cells they visit, and the only way they can sense their current position is to tell when they are next to a wall
    (and which side the wall is on).
    Thus, if $S$ is their finite set of states,
    then their transition function can be written $F: S \times \{\text{wall}, \text{not wall} \}^4 \to S \times \{\uparrow, \downarrow, \leftarrow, \rightarrow\}$,


    Sketch a description of a two-dimensional automaton that can determine
    whether the region it is trapped in is a rectangle.
    \end{quote}
    No need to write the transition function;
    just describe the algorithm and argue it requires only a finite number of states no matter how large the region.

\section*{Solution}

We shall give a highly abstracted algorithm in programming language Rust:

\begin{lstlisting}[language=Rust, style=boxed]

fn verify_rect(env: Environment) -> bool {
    // Initial state: in the middle of nowhere, 
    // keep going to north to find the first wall.
    while !env.wall_on_north() {
        env.step_north();
    }

    // We have find the north wall, goes to north east corner.
    while !env.wall_on_east() {
        env.step_east();

        // If suddenly the north side has no wall, then the north wall is  
        // not straight. e.g. P is our previous position and N is our 
        // current position.
        //     xxxx
        // xxxxx..xxxx
        // x...PN....x
        // x.........x
        // xxxxxxxxxxx
        //
        if !env.wall_on_north() {
            return false;
        }
    }

    // We are in the north east corner, go south.
    while !env.wall_on_south() {
        env.step_south();

        // The east wall is not straight.
        if !env.wall_on_east() {
            return false;
        }
    }

    // We are in the south east corner, go west.
    while !env.wall_on_west() {
        env.step_west();

        // The south wall is not straight.
        if !env.wall_on_south() {
            return false;
        }
    }

    // We are in the south west corner, go north.
    while !env.wall_on_north() {
        env.step_north();

        // The west wall is not straight.
        if !env.wall_on_west() {
            return false;
        }
    }

    // We are in the north west corner, go east.
    let mut length = 0;
    while !env.wall_on_east() {
        env.step_east();
        length += 1;

        // The north wall is not straight.
        if !env.wall_on_north() {
            return false;
        }
    }

    // Now that we know we are in a rectangle space, 
    // we have to vet that the space inside is all empty.
    let mut direction = Direction::Left;
    while !(env.wall_on_south() && (env.wall_on_east() || env.wall_on_west())) {
        env.step_south();
        let mut len = 1;
        match direction {
            Direction::Left => {
                while !env.wall_on_west() {
                    env.step_west();
                    len += 1;
                }
                direction = Direction::Right;
            }
            Direction::Right => {
                while !env.wall_on_east() {
                    env.step_east();
                    len += 1;
                }
                direction = Direction::Left;
            }
        }
        // The length is differnt than the first line means
        // that there is something blocking inside.
        if len != length {
            return false;
        }
    }
    true
}
\end{lstlisting}

\begin{theorem}
    Algorithm defined above is finite.
\end{theorem}

\begin{proof}
    Since this is a finite region as mentioned. 
    Then each loop must finish in a finite step, thus this function is guaranteed to return.
    The time complexity for parameter would be $O(m)$ where $m$ is the diameter of this region.
    The compilexity for inside would be $O(m^2)$ and the overall complexity is $O(m^2)$
\end{proof}

\end{document}
