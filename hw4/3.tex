\documentclass[11pt]{article}

%\usepackage[nosol]{optional}
\usepackage[sol]{optional}
\usepackage{xsim}
\opt{sol}{\xsimsetup{solution/print = true}}
\opt{nosol}{\xsimsetup{solution/print = false}}
\xsimsetup{exercise/the-counter = \alph{exercise}.}


\newcommand{\N}{\mathbb{N}}
\newcommand{\Z}{\mathbb{Z}}
\renewcommand{\P}{\mathsf{P}}
\newcommand{\NP}{\mathsf{NP}}
\newcommand{\SAT}{\textsc{Sat}}

\usepackage{enumerate,color,comment}
\usepackage{graphicx}

\usepackage{fullpage,amsmath,amssymb}
\usepackage[colorlinks=true,citecolor=blue,linkcolor=blue]{hyperref} % for href links, and also makes \ref and \eqref clickable in the PDF


\usepackage{fancyhdr}
\fancypagestyle{firststyle}
{
   \fancyhf{}
   \fancyhead[C]{Copyright \copyright\ \today, David Doty}
}


\title{Homework 4 \opt{sol}{Solutions} -- ECS 220, Winter 2020}
\date{}
\newtheorem{theorem}{Theorem}
\begin{document}
\maketitle
\thispagestyle{firststyle}
\vspace{-2.0cm}

\newcommand{\II}{\#2}
\newcommand{\III}{\#3}
\newcommand{\V}{\#5}
\newcommand{\VII}{\#7}
\newcommand{\XI}{\#11}
\newcommand{\XIII}{\#13}
\newcommand{\XVII}{\#17}

\section{Programming with Fractions}
    The late John Conway invented the following programming language that is defined by a finite sequence of fractions.
    
    Consider the following example:
    
    \[
    \frac{33}{14} \quad
    \frac{21}{22} \quad
    \frac{13}{7} \quad
    \frac{13}{11} \quad
    \frac{26}{85} \quad
    \frac{34}{65} \quad
    \frac{1}{13} \quad
    \frac{1}{17} \quad
    \frac{10}{3} \quad
    \frac{7}{1}
    \]
    
    We start from the value $n=6=2^1 3^1$. At each step, we find the first fraction $f=\frac{p}{q}$ such that $q$ divides $n$ (i.e., such that $n \cdot f$ is an integer), and then replace $n$ by $n \cdot f$. If there is no such $f$, the program halts.
    
    For this example, the sequence of values produced will include the infinite subsequence $2^1 3^1$, $2^1 3^2$, $2^2 3^3$, $2^3 3^5$, $2^5 3^8$, $2^8 3^{13}$, $\ldots$.
    Whenever $n$ is only divisible by $2$ and $3$, the exponents will be adjacent Fibonacci numbers.
    
    Prove why this program outputs the Fibonacci numbers in this way, or instead find your own program that outputs Fibonacci numbers and prove it works.
    
    \textbf{Hint:} you can view this computation as a special type of counter machine.
    See section 7.6.2 in the text for more examples.

\section*{Solution}

We shall setch this model to a counter machine. 
In this machine we use 7 counter to represent the number of each primes(i.e. $2, 3, 5, 7, 11, 13, 17$) that can be decomposed from each step.
In the following proof, we would use notation $\#x$ to refer to the counter that is counting the number of $x$.
Then the step of switching from $n$ to $n\cdot f$ would become increase counters that are in $p$ and decrease the counters that are in $q$.

In this machine, we add the conditions and assertions to each edge.
In this way, we showed that given input of the form $2^s3^t$, the output put has to be in the same form.
Notice the asserts and branches in the graph that guarantees that at accepting state $\V = \VII = \XI = \XII = \XVII = 0$.

Not let's proof that this machine actually generates a Fib sequence.
Since we already known that the initial case is $s = 1, t = 1$, all we need to proof is that, given input $2^s3^t$, the output would be $2^t3^{s+t}$

\begin{theorem}
    Given input $2^s3^t$, $s < t$, the output would be $2^t3^{s+t}$
\end{theorem}

\begin{proof}

    We shall walk through the machine to proof that.
    \begin{enumerate}
        \item Initial input has counter value $\II = s, \III = t$, all other counters are zero.
        \item We would be looping at state $\frac{10}{3}$ until we have $\V = t, \II = s + t, \III = 0$, i.e. there is no $3$ left. 
        \item At state $\frac{7}{1}$, since $q = 1$, it is always executed once and leave us $\VII = 1$, everything else unchanged.
        \item Then we would be looping between state $\frac{33}{14}$ and state $\frac{21}{22}$. These two states aim to wipe out all the $2$. If $\II$ becomes zero at state $\frac{33}{14}$, then it comes to state $\frac{13}{11}$ with $\XI = 1, \III = s + t$; If $\II$ becomes zero at state $\frac{21}{22}$, then it comes to state $\frac{13}{7}$ with $\VII = 1, \III = s + t$.
        \item Either state $\frac{13}{11}$ or state $\frac{13}{7}$ will only executed once and leave us $\XIII = 1, \III = s + t$. 
        \item Then we would be looping between state $\frac{34}{65}$ and state $\frac{26}{85}$. These two states aim to wipe out all the $5$. If $\V$ becomes zero at state $\frac{34}{65}$, then it comes to state $\frac{1}{17}$ with $\XVII = 1, \II = t, \III = s + t$; If $\V$ becomes zero at state $\frac{26}{85}$, then it comes to state $\frac{1}{13}$ with $\XIII = 1, \II = t, \III = s + t$;
        \item Either state $\frac{1}{17}$ or state $\frac{1}{13}$ will only executed once and leave us $\II = t, \III = s + t$. 
        \item Finally it returns to the accepting state $2^t3^{s+t}$ with $\II = t, \III = s + t$. 
    \end{enumerate}
\end{proof}
 \begin{figure}[h]
        \centering
        \begin{tikzpicture}[shorten >=1pt,node distance=2cm,on grid,auto, semithick,
        el/.style = {inner sep=7pt, align=left, sloped}] 
        %\tikzstyle{every node} = [circle, draw]
        \node [state, scale=2,el,label={[blue,align=left]above:$\#2++$\\$\#3--$}] (2) {$\frac{10}{3}$};
        \node [state, initial above,accepting, scale=2,position=40:1 cm from 2,] (1) {$2^s 3^t$};
        \node [state, scale=2, position=200: 2cm from 2,el,label={[blue,align=left]above:$\#7++$}] (3) {$\frac{7}{1}$};
        \node [state, scale=2, position=-20: 2cm from 2,el,label={[blue,align=left]right:$\#13--$}] (4) {$\frac{1}{13}$};
        \node [state, scale=2, below of=3,el,label={[blue,align=left]left:$\#3++$\\$\#11++$\\ $\#2--$\\$\#7--$}] (5) {$\frac{33}{14}$};
        \node [state, scale=2, position=0:0.9cm from 5,el,label={[blue,align=left]above:$\#3++$\\$\#7++$\\ $\#2--$\\$\#11--$}] (6) {$\frac{21}{22}$};
        \node [state, scale=2, right of=6,el,label={[blue,align=left]right:$\#17--$}] (7) {$\frac{1}{17}$};
        \node [state, scale=2, below of=5,el,label={[blue,align=left]left:$\#13++$\\$\#11--$}] (8) {$\frac{13}{11}$};
        \node [state, scale=2, position=0:0.9cm from 8,el,label={[blue,align=left]left:$\#13++$\\$\#7--$}] (9) {$\frac{13}{7}$};
        \node [state, scale=2, below right of=9,el,label={[blue,align=left]below:$\#2++$\\$\#17++$\\ $\#5--$\\$\#13--$}] (11) {$\frac{34}{65}$};
        \node [state, scale=2, position=0: 2.6cm from 9,el,label={[blue,align=left]right:$\#2++$\\$\#13++$\\ $\#5--$\\$\#17--$}] (10) {$\frac{26}{85}$};
        
       
    \path[->]
        (1)  edge[cfgedge, bend right=10] (2)
        (2)  edge[cfgedge, bend right=10] node[el,above] {$\#3=0$} (3)
        (2)  edge[in=240, out=-60,loop] node[el,above] {$\#3\ne0$} (2)
        (3)  edge[cfgedge] node [el,below] {$\#7=1$} (5)
        (5)  edge[cfgedge,bend left=20] node [el,above] {$\#2\ne0$} (6) 
        (6)  edge[cfgedge,bend left=20] node [el,below] {$\#2\ne0$} (5)
    %    (5)  edge[in=150, out=210,loop] node[el,above] {$\#2\ne0$ \\ $\#7\ne0$} (5)
        (5)  edge[cfgedge] node [el,below] {$\#2=0$ } (8)
        (6)  edge[cfgedge] node [el,above] {$\#2=0$ } (9)
        (8)  edge[cfgedge,bend right=30] node [el,above] {$\#13=1$} (11)
        (9)  edge[cfgedge,bend right=10] node [el,above] {$\#13=1$} (11)
        (11)  edge[cfgedge,bend right=20] node [el,below] {$\#5\ne0$} (10)
        (10)  edge[cfgedge,bend right=20] node [el,below] {$\#5\ne0$} (11)
        (11)  edge[cfgedge] node [el,below] {$\#5=0$} (7)
        (10)  edge[cfgedge] node [el,below] {$\#5=0$} (4)
        (7)  edge[cfgedge]  (1)
        (4)  edge[cfgedge]  (1);
        \end{tikzpicture}
        \caption{Directed graph $G$}
        \label{fig:graph1}
      \end{figure}


\end{document}