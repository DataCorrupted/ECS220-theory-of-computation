\documentclass[11pt]{article}

%\usepackage[nosol]{optional}
\usepackage[sol]{optional}
\usepackage{xsim}
\opt{sol}{\xsimsetup{solution/print = true}}
\opt{nosol}{\xsimsetup{solution/print = false}}
\xsimsetup{exercise/the-counter = \alph{exercise}.}


\newcommand{\N}{\mathbb{N}}
\newcommand{\Z}{\mathbb{Z}}
\renewcommand{\P}{\mathsf{P}}
\newcommand{\NP}{\mathsf{NP}}
\newcommand{\SAT}{\textsc{Sat}}

\usepackage{enumerate,color,comment}
\usepackage{graphicx}

\usepackage{fullpage,amsmath,amssymb}
\usepackage[colorlinks=true,citecolor=blue,linkcolor=blue]{hyperref} % for href links, and also makes \ref and \eqref clickable in the PDF


\usepackage{fancyhdr}
\fancypagestyle{firststyle}
{
   \fancyhf{}
   \fancyhead[C]{Copyright \copyright\ \today, David Doty}
}


\title{Homework 4 \opt{sol}{Solutions} -- ECS 220, Winter 2020}
\date{}
\newtheorem{theorem}{Theorem}
\begin{document}
\maketitle
\thispagestyle{firststyle}
\vspace{-2.0cm}


\section{Programming with Fractions}
    The late John Conway invented the following programming language that is defined by a finite sequence of fractions.
    
    Consider the following example:
    
    \[
    \frac{33}{14} \quad
    \frac{21}{22} \quad
    \frac{13}{7} \quad
    \frac{13}{11} \quad
    \frac{26}{85} \quad
    \frac{34}{65} \quad
    \frac{1}{13} \quad
    \frac{1}{17} \quad
    \frac{10}{3} \quad
    \frac{7}{1}
    \]
    
    We start from the value $n=6=2^1 3^1$. At each step, we find the first fraction $f=\frac{p}{q}$ such that $q$ divides $n$ (i.e., such that $n \cdot f$ is an integer), and then replace $n$ by $n \cdot f$. If there is no such $f$, the program halts.
    
    For this example, the sequence of values produced will include the infinite subsequence $2^1 3^1$, $2^1 3^2$, $2^3 3^5$, $2^5 3^8$, $2^8 3^{13}$, $\ldots$.
    Whenever $n$ is only divisible by $2$ and $3$, the exponents will be adjacent Fibonacci numbers.
    
    Prove why this program outputs the Fibonacci numbers in this way, or instead find your own program that outputs Fibonacci numbers and prove it works.
    
    \textbf{Hint:} you can view this computation as a special type of counter machine.
    See section 7.6.2 in the text for more examples.

\end{document}