\documentclass[11pt]{article}
\usepackage{listings, ../listings-rust/listings-rust}
\usepackage{xcolor}
\usepackage{algorithm}
\usepackage[noend]{algpseudocode} % pseudocode
\usepackage{amsmath,amsthm,amssymb}
\usepackage {tikz}
\usetikzlibrary{automata, positioning}
\definecolor{codegreen}{rgb}{0,0.6,0}
\definecolor{codegray}{rgb}{0.5,0.5,0.5}
\definecolor{codepurple}{rgb}{0.58,0,0.82}
\definecolor{backcolour}{rgb}{0.95,0.95,0.92}
\lstset{
    language=Python,
    keepspaces=true,
    numbers=left,
    backgroundcolor=\color{backcolour},
    commentstyle=\color{codegreen},
    basicstyle=\ttfamily,
    otherkeywords={self,True,False,yield},
    keywordstyle=\ttfamily\color{blue!90!black},
    %basicstyle=\footnotesize,
    keywords=[3]{ttk},
    keywordstyle={[2]\ttfamily\color{orange!80!orange}},
    keywordstyle={[3]\ttfamily\color{red!80!orange}},
    emph={MyClass,__init__},
    emphstyle=\ttfamily\color{red!80!black},
    stringstyle=\color{green!80!black},
    showstringspaces=false
}
\usepackage{enumerate}
\usepackage{fullpage,amsmath,amssymb,graphicx}
\usepackage[colorlinks=true,citecolor=blue,linkcolor=blue]{hyperref} % for href links, and also makes \ref and \eqref clickable in the PDF
\newcommand{\QuasiP}{\mathsf{QuasiP}}
\newcommand{\TIME}{\mathsf{TIME}}
\newcommand{\N}{\mathbb{N}}
\newcommand{\Z}{\mathbb{Z}}
\renewcommand{\P}{\mathsf{P}}
\newcommand{\NP}{\mathsf{NP}}
\newcommand{\poly}{\mathrm{poly}}
\newcommand{\SAT}{\textsc{Sat}}
\newcommand{\sat}{\SAT}
\newcommand{\dnfsat}{\textsc{DNF-\sat}}
\newcommand{\cnfsat}{\textsc{CNF-\sat}}
\newcommand{\pmat}{\textsc{Perfect-Matching}}
\newcommand{\iset}{\textsc{Independent-Set}}
\newcommand{\cpc}{\textsc{Cellphone-Capacity}}
\usepackage{fancyhdr}
\fancypagestyle{firststyle}
{
   \fancyhf{}
   \fancyhead[C]{Copyright \copyright\ \today, David Doty}
}
\title{Homework 3 -- ECS 220, Winter 2020}
\date{}
\begin{document}
\maketitle
\thispagestyle{firststyle}
\vspace{-2.0cm}

\section{Search is easy if decision is easy.}
    In computational complexity theory, we study mostly decision problems
    (e.g., \emph{determine if a Boolean formula has a satisfying assignment}),
    instead of the search problems that are typically what practicing computer scientists really want to solve
    (e.g., \emph{determine if a Boolean formula has a satisfying assignment, and then output the assignment}).
    It is natural to wonder whether we are studying the wrong thing.
    In this exercise, you will show that the difficulty of decision problems is in fact linked closely to that of search problems.
        Show that if $\P = \NP$, then every $\NP$ search problem can be solved in polynomial time.
    In other words, for each language $A \in \NP$, with a polynomial-time verifier algorithm $V_A$ taking inputs $x,w$, where $x \in A \iff (\exists w)\ V_A(x,w)$ accepts, then there is a polynomial-time algorithm $S$ that, on input $x$, does the following.
    If $x \not\in A$, then $S(x)$ outputs ``no''.
    If $x \in A$, then $S(x)$ outputs $w$ such that $V_A(x,w)$ accepts.
        {\bf Hint:}
    Build up the witness $w$ bit by bit, by asking $\NP$ questions about the next bit, which are answerable in polynomial time if $\P = \NP$.
    These $\NP$ questions may not exactly correspond to $A$ itself, but a related $\NP$ problem can be defined.


\section*{solution}
Let's consider an algorithm like this.
\begin{algorithm}
\caption{$S(x)$} 
\begin{algorithmic}[]
\Require $x, A, V_A, t$
\If {$x \notin A$} \Return {$No$}
\EndIf
\State {$w = 1$}
\While {$!T(V_A, x, w)$}
\State {$w++$}
\EndWhile
\Return {$w$}
\end{algorithmic}
\end{algorithm}

\begin{lstlisting}
/// Gien a verifier and an input x
/// ask if w is the prefix of the witness.
fn is_prefix(v_a: Verifier, x: Bits, w: Bits) -> bool {
	// prove this is an NP algorithm
	// This is not hard as I can construct a Non-deterministic Turing
	// Machine that randomly guesses the following bits and run the 
	// verifier.
	// P.S. I have no idea how to prove without Non-deterministic 
	// TM.
}

// Given a verifier and an input, find the witness.
// Suppose we already know that the input is in language A.
// The length of `w` must be in Ploy(n), where n is the length of x.
// The function `is_prefix()` is NP, and since NP = P, it is also in
// Ploy(n). Therefore the whole algorithm runs in 
// Ploy(n) * Ploy(n) = Ploy(n).
fn find_witness(v_a: Verifier, x: Bits) -> Bits {
	let mut w = 0
	loop {
		if is_prefix(v_a, x, (w << 1) + 1) {
			w = (w << 1) + 1;
		} else if is_prefix(v_a, x, (w << 1) + 0)  {
			w = (w << 1) + 0;
		} else {
            // If add 0 and 1 cannot be prefix
            // then we must have found the witness
			return w;
		}
	}
}
\end{lstlisting}

Now let's take a look at the algorithm here. Denote the desired witness to be 
$w_0$. $is\_prefix()$ is the function to determind whether current $w$ is the 
prefix of $w_0$. $find\_witness()$ is the function that we build up $w$ bit by 
bit and check if the \emph{next} $w$ is $w_0$. If it's not, we keep building it 
up. If it is, we return $w$.

Note that this is the function that handles the situation when $x \in A$. In the 
whole process we need to first determine whether $x$ is in $A$, that should be 
done in $Poly(n)$ since the size of $x$ is $n$. 

$is\_prefix()$


Since we can find $w$ in polynomial time, we prove that we can solve the 
search problem in polynomial time if $P = NP$.

\end{document}