\documentclass[11pt]{article}
\usepackage{listings}
\usepackage{xcolor}
\definecolor{codegreen}{rgb}{0,0.6,0}
\definecolor{codegray}{rgb}{0.5,0.5,0.5}
\definecolor{codepurple}{rgb}{0.58,0,0.82}
\definecolor{backcolour}{rgb}{0.95,0.95,0.92}
\lstset{
    language=Python,
    keepspaces=true,
    numbers=left,
    backgroundcolor=\color{backcolour},
    commentstyle=\color{codegreen},
    basicstyle=\ttfamily,
    otherkeywords={self,True,False,yield},
    keywordstyle=\ttfamily\color{blue!90!black},
    %basicstyle=\footnotesize,
    keywords=[3]{ttk},
    keywordstyle={[2]\ttfamily\color{orange!80!orange}},
    keywordstyle={[3]\ttfamily\color{red!80!orange}},
    emph={MyClass,__init__},
    emphstyle=\ttfamily\color{red!80!black},
    stringstyle=\color{green!80!black},
    showstringspaces=false
}
\usepackage{enumerate}
\usepackage{fullpage,amsmath,amssymb,graphicx}
\usepackage[colorlinks=true,citecolor=blue,linkcolor=blue]{hyperref} % for href links, and also makes \ref and \eqref clickable in the PDF
\newcommand{\QuasiP}{\mathsf{QuasiP}}
\newcommand{\TIME}{\mathsf{TIME}}
\newcommand{\N}{\mathbb{N}}
\newcommand{\Z}{\mathbb{Z}}
\renewcommand{\P}{\mathsf{P}}
\newcommand{\NP}{\mathsf{NP}}
\newcommand{\poly}{\mathrm{poly}}
\newcommand{\SAT}{\textsc{Sat}}
\newcommand{\sat}{\SAT}
\newcommand{\dnfsat}{\textsc{DNF-\sat}}
\newcommand{\cnfsat}{\textsc{CNF-\sat}}
\newcommand{\pmat}{\textsc{Perfect-Matching}}
\newcommand{\iset}{\textsc{Independent-Set}}
\newcommand{\cpc}{\textsc{Cellphone-Capacity}}
\usepackage{fancyhdr}
\fancypagestyle{firststyle}
{
   \fancyhf{}
   \fancyhead[C]{Copyright \copyright\ \today, David Doty}
}
\title{Homework 3 -- ECS 220, Winter 2020}
\date{}
\begin{document}
\maketitle
\thispagestyle{firststyle}
\vspace{-2.0cm}

\section{Search is easy if decision is easy.}
    In computational complexity theory, we study mostly decision problems
    (e.g., \emph{determine if a Boolean formula has a satisfying assignment}),
    instead of the search problems that are typically what practicing computer scientists really want to solve
    (e.g., \emph{determine if a Boolean formula has a satisfying assignment, and then output the assignment}).
    It is natural to wonder whether we are studying the wrong thing.
    In this exercise, you will show that the difficulty of decision problems is in fact linked closely to that of search problems.
        Show that if $\P = \NP$, then every $\NP$ search problem can be solved in polynomial time.
    In other words, for each language $A \in \NP$, with a polynomial-time verifier algorithm $V_A$ taking inputs $x,w$, where $x \in A \iff (\exists w)\ V_A(x,w)$ accepts, then there is a polynomial-time algorithm $S$ that, on input $x$, does the following.
    If $x \not\in A$, then $S(x)$ outputs ``no''.
    If $x \in A$, then $S(x)$ outputs $w$ such that $V_A(x,w)$ accepts.
        {\bf Hint:}
    Build up the witness $w$ bit by bit, by asking $\NP$ questions about the next bit, which are answerable in polynomial time if $\P = \NP$.
    These $\NP$ questions may not exactly correspond to $A$ itself, but a related $\NP$ problem can be defined.


\section*{solution}
Let's consider an algorithm like this.
\begin{algorithm}
\caption{$S(x)$} 
\begin{algorithmic}[]
\Require $x, A, V_A, t$
\If {$x \notin A$} \Return {$No$}
\EndIf
\State {$w = 1$}
\While {$!T(V_A, x, w)$}
\State {$w++$}
\EndWhile
\Return {$w$}
\end{algorithmic}
\end{algorithm}

This algorithm takes input $x$, language $A$, the polynomial-time verifier 
algorithm $V_A$ and a time bound $t$, which is in unary. First it verifies 
whether $x \in A$, if not, returns $No$. Denote the size of $x$ to be $n$, 
the verification process of whether $x \in A$ should be done in $Poly(n)$. 
If $x \in A$, the algorithm tries to build up the witness $w$ bit by bit 
and repeatedly tests whether $V_A$ can verify input $x$ with witness $w$ 
in polynomial time. The test can be done by plugging in $w$ and run $V_A$ 
to see if $V_A$ can do it within the time bound $t$, which ensures the test 
to be in $Poly(n)$. If yes, we find the $w$ we need, if no, we continue to 
try next $w$. Since $w$ has a size of $Poly(n)$ and $T(V_A, x, w)$ is in 
$Poly(n)$, $S(x)$ has a time complexity of $Poly(n) + Poly(n)*Poly(n) = Poly(n)$. 
Since $x \in A \iff (\exists w)V_A(x, w)$, 
we are ensured to find $w$ in polynomial time, which means we can solve the 
search problem in polynomial time if $P = NP$.

\end{document}