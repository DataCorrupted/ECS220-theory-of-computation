\documentclass[11pt]{article}

%\usepackage[nosol]{optional}
\usepackage[sol]{optional}

\usepackage {tikz}
    \usetikzlibrary{calc,shapes,arrows,automata,positioning,cd}
    \tikzset{
     dot node/.style={font=\sffamily\small},
      cfgedge/.style   = {black, ->, >=stealth},
      forward/.style = { blue, ->, >=angle 45},
      backward/.style = { red, densely dashed, ->, >=latex' },
      backwardleft/.style = { red, densely dashed, <-, >=latex' },
      position/.style args={#1:#2 from #3}{
        at=(#3.#1), anchor=#1+180, shift=(#1:#2)
    },
    }
    \usepackage{xcolor}

\usepackage{listings, ../listings-rust/listings-rust}
\usepackage{listings}
\usepackage{xcolor}
\definecolor{codegreen}{rgb}{0,0.6,0}
\definecolor{codegray}{rgb}{0.5,0.5,0.5}
\definecolor{codepurple}{rgb}{0.58,0,0.82}
\definecolor{backcolour}{rgb}{0.95,0.95,0.92}
\lstset{
    language=Python,
    keepspaces=true,
    numbers=left,
    backgroundcolor=\color{backcolour},
    commentstyle=\color{codegreen},
    basicstyle=\ttfamily,
    otherkeywords={self,True,False,yield},
    keywordstyle=\ttfamily\color{blue!90!black},
    %basicstyle=\footnotesize,
    keywords=[3]{ttk},
    keywordstyle={[2]\ttfamily\color{orange!80!orange}},
    keywordstyle={[3]\ttfamily\color{red!80!orange}},
    emph={MyClass,__init__},
    emphstyle=\ttfamily\color{red!80!black},
    stringstyle=\color{green!80!black},
    showstringspaces=false
}

\newcommand{\N}{\mathbb{N}}
\newcommand{\Z}{\mathbb{Z}}

% For proof.
\usepackage{amsmath,amsthm,amssymb}

\usepackage{enumerate}
\usepackage{graphicx}
\usepackage{float}
\usepackage{subcaption}
\usepackage{comment}

\renewcommand{\topfraction}{.9}
\renewcommand{\textfraction}{.1}

\usepackage{fullpage,amsmath,amssymb}
\usepackage[colorlinks=true,citecolor=blue,linkcolor=blue]{hyperref} % for href links, and also makes \ref and \eqref clickable in the PDF

% parenthetical comment to state verbally but not write on the board
\newcommand{\com}[1]{\footnote{#1}}

\newcommand{\deltahat}{\widehat{\delta}}

\newcommand{\QuasiP}{\mathsf{QuasiP}}

\newcommand{\TIME}{\mathsf{TIME}}

\usepackage{fancyhdr}
\fancypagestyle{firststyle}
{
    \fancyhf{}
    \fancyhead[C]{Copyright \copyright\ \today, David Doty}
}


\title{Homework 5 \opt{sol}{Solutions} -- ECS 220, Winter 2020}
\date{}
\begin{document}
\maketitle
\thispagestyle{firststyle}
\vspace{-2.0cm}


\section*{Matched parentheses}
Suppose we are given a string of parentheses.
First show that telling whether they are properly matched and nested,
such as $(()())$ or $((()))()$ but not $())$ or $())(()$,
is in $\mathsf{L}$.

Now suppose there are two types of parentheses, round ones and square brackets.
We wish to know if they are properly nested: for instance,
$([])[]$ and $[()[]]$ are allowed,
but $([)]$ is not.
Show that this problem is in $\mathsf{L}$.

    {\bf Hint:} Each opening parentheses ( or [ has a ``closing partner'' ) or ].
Ensure that the type of each opening is the same as the type of its closing partner.


\section*{Solution}

(Source code is listed \href{https://github.com/DataCorrupted/parentheses-matching}{here}.
You will need \href{https://www.rust-lang.org/}{Rust programming language} to run it.
Download it and \lstinline{cargo test} to run it.)

\subsection*{Single parentheses}

Solving this is relatively easy, all we need is a counter \lstinline{diff} to keep record of the difference between left parenthese $($ and right ones.
If \lstinline{diff} is zero then the string happen to be matching.
If smaller than zero then we have more right parenthese than left ones.

It's easy to argue that this program is in $L$ as only one counter \lstinline{diff} is used and it will not exceed $n$, thus only uses $log n$ space.

\begin{lstlisting}[language=Rust]
/// Test if strings only consists of '(' and ')' matches.
fn paren_matching<F>(s: &String, is_left: F) -> bool
where
    F: Fn(char) -> bool,
{
    let mut diff = 0;
    for c in s.chars() {
        if is_left(c) {
            diff += 1;
        } else {
            diff -= 1;
        }
        if diff < 0 {
            return false;
        }
    }
    diff == 0
}
fn one_paren(c: char) -> bool {
    c == '('
}
fn main() {
    assert!(paren_matching(&"()()()".to_owned(), one_paren));
    assert!(paren_matching(&"((()))".to_owned(), one_paren));
    assert!(!paren_matching(&"())".to_owned(), one_paren));
}
\end{lstlisting}

\subsection*{Two parentheses}

We would start by verify if the string matches if we ignore the type of the parentheses, i.e. we replace $[$ with $($, $]$ with $)$.
This first step eliminates some errors like $)($ or $()]$, i.e. left/right parenthese number mismatch and/or place mismatch.

Now we only have to deal with the type, as we know it's a perfectly matches if we don't consider the parenthese type.
We observe one important fact that, \textbf{for two matching parentheses, the number of bytes inside must be odd.}
We would use this to determine if there is a mismatch, i.e. $([)]$ will be detected by this property.

While finding matching right parenthese, we also (accidentally) made sure that the number of two type of the parenthese must also match.
For example, if the number mismatch, $([]]$, first char $($ will not find it's matching right parenthese.

The code will be listed below.

\begin{lstlisting}[language=Rust]
/// Finds the paren that matches the left paren.
///
/// If found, the index is returned, else None to indicate that
/// there is no matching one for this char, which is a clear
/// indicator that this string does not match.
fn find_matching_paren(vec: &Vec<char>, idx: usize) -> Option<usize> {
    let target = if vec[idx] == '(' { ')' } else { ']' };
    let mut diff = 0;
    for i in idx..vec.len() {
        let curr = vec[i];
        if curr == vec[idx] {
            diff += 1;
        } else if curr == target {
            diff -= 1;
        }
        if diff < 0 {
            return None;
        }
        if diff == 0 {
            return Some(i);
        }
    }
    None
}

fn two_parens(c: char) -> bool {
    c == '(' || c == '['
}

/// Test if strings only consists of '(', ')', '[' and ']' matches.
fn parens_matching(s: &String) -> bool {
    if !paren_matching(s, two_parens) {
        return false;
    }
    let vec: Vec<char> = s.chars().collect();
    for i in 0..s.len() {
        if !two_parens(vec[i]) {
            continue;
        }
        if let Some(right_idx) = find_matching_paren(&vec, i) {
            println!("{} matching paren {}", i, right_idx);
            if (right_idx - i + 1) & 0x1 == 1 {
                return false;
            }
        } else {
            return false;
        }
    }
    true
}

fn main() {
    assert!(parens_matching(&"([][])()".to_owned()));
    assert!(!parens_matching(&"(()]()".to_owned()));
    assert!(!parens_matching(&"([()()[()])]".to_owned()));
}
\end{lstlisting}

It is also easy argue that this algorithm is in $L$, as there is only two counter involved:
the index of current char, and the index of the char that matches current char.
These two indexs both smaller than $n$, and thus takes $log n$ space.

\end{document}
